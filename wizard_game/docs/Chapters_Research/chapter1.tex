\chapter{Background \& Objectives}


\section{Background}
In this section I will be talking about the Wizard Card Game, how the game works and how you can win the game. I will also discuss the Artificial Intelligence techniques that were consider using for the opponents and the Monte Carlo tree search that I will be using for the game.
\subsection{Wizard Card Game}
\subsubsection{Dealing and Setup}
Wizard is a card game that contains 52 normal deck cards and 8 extra cards 4 of which are called wizards and the other 4 are called jesters.  Each player is dealt the same number of cards that is the round. For example, round 1 means each person get 1 card but round 10 means they each get ten cards. This is then played until the is no cards left to be dealt out mean for 3 people there should be 20 round and 15 rounds for 4 people.  A trump card is also dealt from the deck at the beginning of the round from the top.  After each play in a round the dealer is changed clockwise.
\subsubsection{Playing and Bidding}
One the trump has been dealt, the person to the left of the dealer states how many tricks they think they will win in this round. The maximum they can bid it the number of cards they have in their hands.  Once everyone has put their individual bids in, the player to the left of the dealer puts down the first card which can be any card they want, then each player can either match the suit that the first player put down or the better play it to match the trump suit or play a wizard to win. Otherwise they can play off suit or play a jester to lose.
\subsubsection{Scoring}
To score points in this game you must use the bid on how many tricks you think you can win, giving you 20 point for being right and 10 point for every trick you win. Otherwise you get negative 10 point per every trick you were over or under for that round.  For example, if you guess 4 tricks to win and only one 1 you would lose 30 points
\subsection {Monte Carlo Alogrithm}

\subsection{Similar Algorithms}

\subsubsection{Neural Network}

\subsubsection{MinMax}


\section{Analysis}
In this section I will be talking about what I will be aiming doing achieve with this product and what technology I will be using during this process. 
\subsection{Project Aims}

\subsection {Technology Overview}
\subsection{ Data Selection}

\section{Research Method and Software Process}
\subsection {Foreseen Challenges}
\subsection {Process Methodology}